\def\year{2021}\relax
%File: formatting-instructions-latex-2021.tex
%release 2021.1
\documentclass[letterpaper]{article} % DO NOT CHANGE THIS
\usepackage{aaai21}  % DO NOT CHANGE THIS
\usepackage{times}  % DO NOT CHANGE THIS
\usepackage{helvet} % DO NOT CHANGE THIS
\usepackage{courier}  % DO NOT CHANGE THIS
\usepackage[hyphens]{url}  % DO NOT CHANGE THIS
\usepackage{graphicx} % DO NOT CHANGE THIS
\urlstyle{rm} % DO NOT CHANGE THIS
\def\UrlFont{\rm}  % DO NOT CHANGE THIS
\usepackage{natbib}  % DO NOT CHANGE THIS AND DO NOT ADD ANY OPTIONS TO IT
\usepackage{caption} % DO NOT CHANGE THIS AND DO NOT ADD ANY OPTIONS TO IT
\frenchspacing  % DO NOT CHANGE THIS
\setlength{\pdfpagewidth}{8.5in}  % DO NOT CHANGE THIS
\setlength{\pdfpageheight}{11in}  % DO NOT CHANGE THIS

\usepackage{amssymb}
\usepackage{amsmath}
% checkmarks
\usepackage{pifont}
\newcommand{\cmark}{\ding{51}}
\newcommand{\xmark}{\ding{55}}


\setcounter{secnumdepth}{2} %May be changed to 1 or 2 if section numbers are desired.



%This is a template for producing LIPIcs articles.
%See lipics-manual.pdf for further information.
%for A4 paper format use option "a4paper", for US-letter use option "letterpaper"
%for british hyphenation rules use option "UKenglish", for american hyphenation rules use option "USenglish"
%for section-numbered lemmas etc., use "numberwithinsect"
%for enabling cleveref support, use "cleveref"
%for enabling cleveref support, use "autoref"


\usepackage{booktabs}
\usepackage{multirow}
\usepackage[algo2e,vlined]{algorithm2e}

\pdfinfo{
/Title (A Fast and Tight Heuristic for A* in Road Networks)
/Author (Foo Bar)
/Keywords (shortest path, road graphs, goal-directed search, contraction hierarchy, heuristic search)
}
\title{A Fast and Tight Heuristic for A* in Road Networks}
\author{Foo Bar}


% \author{Ben Strasser\\
% academia@ben-strasser.net\\
% \And
% Tim Zeitz\\
% tim.zeitz@kit.edu\\
% Karlsruhe Institute of Technology
% }

%%
%% The "author" command and its associated commands are used to define
%% the authors and their affiliations.
%% Of note is the shared affiliation of the first two authors, and the
%% "authornote" and "authornotemark" commands
%% used to denote shared contribution to the research.
%\author{
%Ben Strasser


%\author{Tim Zeitz}
%tim.zeitz@kit.edu}
%\affiliation{%
%  \institution{Institute of Theoretical Informatics, Algorithmics I, Karlsruhe Institute of Technology}
%  \city{Karlsruhe}
%  \country{Germany}
%}


\begin{document}

\maketitle

%%
%% By default, the full list of authors will be used in the page
%% headers. Often, this list is too long, and will overlap
%% other information printed in the page headers. This command allows
%% the author to define a more concise list
%% of authors' names for this purpose.
% \renewcommand{\shortauthors}{Strasser and Zeitz}

%%
%% The abstract is a short summary of the work to be presented in the
%% article.
\begin{abstract}
Efficiently computing flexible and exact routes in large road networks is a tough challenge.
While Dijkstra's algorithm can be used to solve these problems, it is too slow for many practical applications.
A* is a classical approach to accelerate Dijkstra's algorithm and is used in many fields such as Experimental Algorithms, Robotics, and Artificial Intelligence.
However, A*'s performance depends on the availability of a good heuristic.
Computing a good heuristic is a challenge of its own.
In road networks, shortest paths can alterantively be quickly computed using hierarchical techniques.
They are not based on A* and do not use a heuristic.
Hierarchical techniques use a slow preprocessing and fast query phase.
The preprocessing phase computes auxiliary data that is used in the query phase.
The auxiliary is independent of origin or destination of the search.
% Hierarchical techniques are known to work well on road networks.
Hierarchical techniques achieve speed and exactness but sacrifice A*'s flexibility.
In this paper, we use Contraction Hierarchies (CH), a hierarchical technique, to compute preprocessing-tight A* heuristics.
We call our technique CH-Potentials.
With this approach, we combine some of the speed of hierarchical techniques with the flexibility offered by A* - the best of both worlds - while retaining exactness.
Additionally, we describe A* optimizations to accelerate the processing of low degree nodes, which often occur in road networks.
\end{abstract}

\section{Introduction}
\label{sec:intro}
Route planning in large street networks is a well-researched topic with many applications~\cite{bdgmpsww-rptn-16}.
It is often formalized as a variant of the classical shortest path problem in weighted graphs.
Developing flexible techniques to efficiently compute exact routes is a tough challenge.
\emph{Exact} means that a path of provable minimum weight is computed.
\emph{Efficiently} means that the path computation is fast.
Being able to change the edge weights between path computations or handling complex weights, such as edge weight functions, is \emph{flexibility}. % finde es klingt stringenter, wenn jede Eigenschaft genau einen Satz hat.

\begin{figure}

\centering
\includegraphics[width=\columnwidth]{fig/searchspace_st.png}


\caption{Nodes explored by A* with CH-Potentials. Color indicates the node removal order from the queue. Blue was removed first. Next is green. Red was removed last.}
\label{img:search-space}
\end{figure}


Dijkstra's algorithm~\cite{d-ntpcg-59} achieves flexibility and exactness but is comparatively slow.
It iteratively explores nodes ordered by their distance from the start.
A classic speedup approach is the well-known A* algorithm~\cite{hnr-afbhd-68}.
It processes nodes in a different order, preferring nodes closer to the destination.
Which nodes are considered close is encoded using a heuristic function.
While flexible, A*'s performance crucially depends on the quality of this function.
Finding good heuristics that can be evaluated efficiently is difficult.
Constructing good heuristics on road networks, such as the one depicted in Figure~\ref{img:search-space}, is the focus of this paper.

In this paper, we consider a setup with an offline preprocessing phase and an online query phase.
The preprocessing phase can be slow and computes auxiliary data that can be used in the query phase, which should be fast.
The input to the preprocessing phase is a directed graph $G=(V,E)$ weighted with \emph{lower bound weights} $w_\ell$.
The query phase is given the \emph{source} $s$ and the \emph{target} $t$ as input.
$s$ is also called start and synonyms for $t$ are goal or destination.
It is further given the \emph{query edge weights} $w_q$.
The task is to compute a minimum length path with respect to $w_q$.
Of $w_q$, it is only known that $w_\ell$ is a lower bound, i.e., $w_\ell \le w_q$.
%
While the preprocessing phase may be slow, it must eventually finish.
On graphs with millions of nodes, quadratic preprocessing time is prohibitive.

In this paper, we consider an algorithm, where the query phase mostly consists of an A* search.
In the preprocessing phase, it computes a Contraction Hierarchy (CH)~\cite{gssv-erlrn-12}.
The CH is used to efficiently evaluate the heuristic.
Our setup is very similar to the well-known ALT technique~\cite{gh-cspas-05,DBLP:conf/wea/DellingW07}.
Following the ALT terminology, we call our algorithm \emph{CH-Potentials}, as it combines ideas from Contraction Hierarchies (CH)~\cite{gssv-erlrn-12} with A* heuristics.

A heuristic $h$ is a function that maps nodes onto numbers.
$h$ depends on the target $t$ but not on the source $s$.
Without loss of generality, we assume that $h(t)=0$.
The \emph{tight} heuristic assigns to every node $x$, the length of a shortest $xt$-path.
All heuristics, for which A* is exact, are lower bounds of the tight heuristic.
CH-Potentials compute the tight heuristic with respect to $w_\ell$.
Without access to the query weights $w_q$, which are unavailable during preprocessing, no value of the heuristic can be increased, without loosing A*'s exactness.
In this sense, CH-Potentials are as tight as possible and thus optimal.
We say that a heuristic that is tight with respect to $w_\ell$ is \emph{preprocessing-tight}.

\section{Related Work}

A lot of research focuses on the inflexible setting, i.e., the special case where $w_q = w_\ell$.
In this setting, mostly hierarchical approaches dominate in terms of running time~\cite{bdgmpsww-rptn-16}.
A well-known hierarchical technique are Contraction Hierarchies (CH)~\cite{gssv-erlrn-12}.

There is a lot of work that extends CH to handle more complex settings.
For example, in~\cite{fns-opca-14,gks-rpfof-10} multi-criteria optimization is studied.
In~\cite{dsw-cch-15}, CH is modified to support Live-Traffic.
Historic traffic patterns are combined with CH in~\cite{swz-sfert-19,bgsv-mtdtt-13,bdpw-dtdrp-16}.
Specialized variants for electric vehicle routing are developed in~\cite{DBLP:journals/algorithmica/BaumDPSWZ20,DBLP:conf/aaai/EisnerFS11}.
While these works show that it is possible to extend hierarchical approaches, they also show that it is non-trivial.
Further, in every setting the flexibility available at query time is fairly limited.
Combining all these hierarchical extensions is an unsolved problem.
%
CH-Potentials is not the first work to combine hierarchical approaches and A*~\cite{bdsssw-chgds-10,gkw-blwr-07,bdgwz-sfpcs-19}.
However, a lot of past works use A* to accelerate hierarchical approaches even further and loose A*'s' flexibility.

ALT~\cite{gh-cspas-05} and CPD-Heuristics~\cite{DBLP:conf/ijcai/BonoGHS19} are the two techniques with high conceptual similarity to CH-Potentials.
Just as our approach, ALT is A*-based.
However, ALT's heuristic is not preprocessing-tight.
%
CPD-Heuristics are a combination of A* and Compressed Path Databases (CPD).
A CPD can quickly compute the first edge of a shortest path between any two nodes.
In~\cite{DBLP:conf/ijcai/BonoGHS19}, SRC~\cite{DBLP:conf/socs/StrasserHB14} is used as CPD.
For every distance estimation, a shortest path to the target is computed, whose length is used as the heuristic value.
Unfortunately, the employed CPD's quadratic preprocessing running time is prohibitive on large street networks.
%
MtsCopa~\cite{DBLP:journals/tciaig/BaierBHH15} uses a similar idea to CPD-Heuristics but is only applied in a hunter-prey application.
%
In~\cite{DBLP:conf/ijcai/0002UJAKK18} the weighted graph is embedded into Euclidean space using FastMap such that distances in space and distances in the graph roughly correspond.
The Euclidean distance is then used as A* heuristic.
This results in a good but not a preprocessing-tight heuristic.
%
The handling of low degree nodes in our A* search is a variation of the techniques used in TopoCore~\cite{DBLP:conf/gis/DibbeltSW15}, which is inspired by~\cite{DBLP:journals/pvldb/FunkeNS14}.

\subsection{Oracle-A* \& Comparison with Related Work}
\label{sec:oracle}

To compare against related work, we introduce the concept of Oracle-A*.
It has access to a shortest distance array with respect to $w_\ell$, i.e., it is preprocessing-tight.
The array computation time is unaccounted for, i.e., it is magically filled instantly.
Oracle-A* is not achievable in practice.
However, it is a lower bound for every algorithm following our two phase setup.
We experimentally show that CH-Potentials are within a factor two of Oracle-A* and thus every related work.

\subsection{Applications need Flexibility}

Many applications need flexibility.
The most common ones are traffic avoidance, user preferences, and vehicle restrictions.
Traffic avoidance comes in two forms: Live~\cite{dgpw-crprn-13,dsw-cch-15} and predicted traffic~\cite{ndls-bastd-12,bgsv-mtdtt-13}.
Predicted traffic is often called time-dependent routing.
%
User preferences can vary significantly~\cite{DBLP:conf/gis/FunkeS15,DBLP:conf/gis/DellingGGKTW15,DBLP:conf/gis/FunkeLS16}.
For example, for some users shorter routes are more important than fast routes.
Others want to avoid tunnels or highways.
%
Flexibility is also often used as building block in multi-agent planning settings~\cite{DBLP:journals/ai/SharonSFS15,DBLP:journals/tciaig/BaierBHH15}.
%
\section{Contribution \& Outline}

We introduce CH-Potentials, a two-phase technique to efficiently compute preprocessing-tight heuristics, in Section~\ref{sec:main-algo}.
Combined with A*, this yields an exact, efficient, and flexible route planning algorithm.
Beside the heuristic, in Section\ref{sec:low-deg-improvment}, we improve A*'s handling of low degree nodes, common in road networks.
In Section~\ref{sec:extensions}, we demonstrate CH-Potential's flexibility, by describing algorithms for various extended route planning applications.
Finally, in Section~\ref{sec:experiments}, we present an experimental evaluation of our approach.
By comparing against Oracle-A*, we show that no other exact two-phase A*-based approach can be significantly faster.

\section{Algorithm Description}

In this section, we first describe Contraction Hierarchies, then PHAST, a CH extension, and finally CH-Potentials.

\label{sec:main-algo}

\begin{algorithm2e}
\KwData{$B[x]$: tentative distance from $x$ to target $t$}
\KwData{Min. priority queue $Q$, also called open list}
$B[x] \leftarrow +\infty$ for all $x\neq t$;
$B[t] \leftarrow 0$\;
Make $Q$ only contain $t$ with weight $0$\;
\While{not $Q$ empty}{
	$y\leftarrow$ pop minimum element from $Q$\;
	\For{$xy$ is down-edge in $G^+$}{
		\If{$B[x] > w_\ell(xy) + B[y]$}{
			$B[x]\leftarrow w_\ell(xy) + B[y]$\;
                        Add $x$ or decrease $x$'s key in $Q$ to $B[x]$\;
		}
	}
}
\caption{CH backward search}
\label{algo:ch-backward}
\end{algorithm2e}

\begin{figure}
\centering
\includegraphics{fig/ch}
\caption{
Solid lines are edges in $G$. Dotted lines are shortcuts. Red is shortest $st$-path in $G$. Blue is equaly long up-down $st$-path in $G^+$. $m$ is the mid node.
}
\label{fig:ch}
\end{figure}

\subsection{Contraction Hierarchy (CH)}

A CH is a two phase technique to efficiently compute exact, shortest paths.
It is not flexible, i.e., requires $w_q=w_\ell$.
For a details, we refer to \cite{gssv-erlrn-12,dsw-cch-15}.
In this section, we give an introduction.

A CH places nodes into levels.
No edge must connect two nodes within one level.
Levels are ordered by ``importance''.
The intuition is that dead-ends are unimportant and at the bottom while highway bridges are very important and at the top.
An edge goes \emph{up} when it goes from a node in a lower level to higher level.
\emph{Down} edges are defined analogously.
An \emph{up-down path} is a path where only one node $m$ is more important than both its neighbors.
$m$ is called the \emph{mid} node.
An \emph{up path} is a path where the last node is the mid node.
Similarly, the first node is the mid node of a \emph{down path}.
% Every up and down path is an up-down path.
%
In the preprocessing phase, a CH adds \emph{shortcut} edges to the input graph $G$ to obtain $G^+$.
This is done by repeatedly contracting unimportant nodes and adding shortcuts between its neighbors.
%See~\cite{gssv-erlrn-12} for the details.
After the preprocessing, for every pair of nodes $s$ and $t$ there exists a shortest up-down $st$-path in $G^+$ with the same length as a shortest path in $G$.
See Figure~\ref{fig:ch} for a proof sketch.
From every shortest path (red) in $G$, an up-down path of equal length in $G+$ (blue) exists.
Thus, we can restrict our search to up-down paths in $G^+$.
The search is bidirectional.
The forward search starts from $s$ and only follows up-edges.
Similarly, the backward search starts at $t$ and only follows down-edges in reversed direction.
The two searches meet at the mid node.
Pseudo-code for the backward search, i.e., the path from $m$ to $t$, is presented in Algorithm~\ref{algo:ch-backward}.
The forward search works analogously.
%
A CH query is fast, if the number of nodes reachable via only up- or down-nodes is small.
On road networks, this is the case~\cite{gssv-erlrn-12,dgrw-gpnc-11,dgpw-crprn-13}.
On graphs with low treewidth, this is also the case~\cite{dsw-cch-15,hs-gbpo-18}.

Using a CH, we can compute a simple and preprocessing-tight heuristic.
In the preprocessing phase, a CH with respect to $w_\ell$ is computed.
The heuristic evaluation $h(x)$ performs a CH-query from $x$ to $t$.
%
This works, however, while one CH query is fast, answering one for every node explored in the $A^*$ search is slow.
Fortunately, we can do better.
However, for this we need another component called PHAST~\cite{dgnw-phast-13}.

\subsection{PHAST based Heuristic}

\begin{algorithm2e}
\KwData{$P[x]$: tentative distance from $x$ to $t$}
Execute Algorithm~\ref{algo:ch-backward}\;
\For{all CH levels $L$ from most to least important}{
	\For{all up edges $xy$ in $G^+$ with $x$ in $L$}{
		\If{$P[x] < P[y] + w_\ell(xy)$}{
			$P[x] \leftarrow P[y] + w_\ell(xy)$\;
		}
	}
}
\caption{PHAST basic all-to-one search}
\label{algo:phast}
\end{algorithm2e}

PHAST~\cite{dgnw-phast-13} is a CH extension that computes distances from all nodes to one target node.
First the preprocessing phase is executed analogously to the original CH.
The query phase is divided into two steps.
The first step is analogue to the CH query:
From $t$, all reachable nodes via reversed down-edges are explored.
Algorithm~\ref{algo:ch-backward} shows this first step.
The second step iterates over all CH levels from top to bottom.
In each iteration, all up-edges starting within the current level are followed in reverse.
After all levels are processed, the distances from all nodes towards $t$ is computed.
%PHAST is slightly faster than Dijkstra's algorithm on road graphs because it is a better fit for modern processor architectures.
%PHAST's main advantage is that can easily be parallelized.
%However, we will not consider parallelization in this paper.
%We refer to~\cite{dgnw-phast-13} for an in-depth experimental performance analysis.
Pseudo-code is provided in Algorithm~\ref{algo:phast}.
Using PHAST, we can also compute a preprocessing-tight A* heuristic.
In the query phase, we first run PHAST to compute the distances from every node to $t$ with respect to $w_\ell$ and store the result in an array $H$.
Next, we run $A^*$ and implement the heuristic as a lookup in the array $H$.

% This PHAST-based algorithm works.
The $H$ lookup and by extension the $A^*$ search is indeed fast.
However, the PHAST step before the search is comparatively expensive.
The reason is that the distances towards $t$ are computed for \emph{all} nodes.
Ideally, we only want to compute the distances from the nodes explored in the $A^*$ search.

\subsection{CH-Potentials}

\begin{algorithm2e}
\KwData{$B[x]$: tentative distance from $x$ to $t$ as computed by Algorithm~\ref{algo:ch-backward}}
\KwData{$P[x]$: memoized potential at $x$, $\bot$ initially}
\SetKwFunction{Pot}{Pot}
\SetKwProg{Fn}{Function}{:}{}
\Fn{\Pot{$x$}}{
	\If{$P[x] = \bot$}{
		$P[x]\leftarrow B[x]$\;
		\For{all up edges $xy$ in $G^+$}{
                        $P[x]\leftarrow\min\{P[x],w_\ell(xy)+\Pot(y)\}$\;
		}
	}
	\Return{$P[x]$}\;
}
\caption{CH-Potentials Algorithm}
\label{algo:pot}
\end{algorithm2e}

Fortunately, the PHAST computation can be done lazily using memoization as depicted in Algorithm~\ref{algo:pot}.
In a first step, we run the backward CH search from $t$ to obtain an array $B$.
$B[x]$ is the minimum down $xt$-path distance or $+\infty$, if there is no such path.
$B$ is computed as shown in Algorithm~\ref{algo:ch-backward}.

To compute the heuristic $h(x)$, we recursively compute for all up-edges $(x,y)$ the heuristic $h(y)$.
Next, we compute the minimum distance over all up-down paths that contain at least one up-edge using $d = \min_y\{w_\ell(x,y) + h(y)\}$.
As not all shortest up-down paths contain an up-edge, we set $h(x) = \min \{ B[x], d \}$.
This calculation is correct, as it computes the minimum up-down $xt$-path distance in $G^+$, which corresponds to the minimum $xt$-path distance in a CH.
A* with this heuristic is the basic CH-Potentials algorithm.

\section{Low Degree A* Improvements}

\label{sec:low-deg-improvment}

Preliminary experiments showed, that most A* running time is spent in heuristic evaluations and queue operations.
We can reduce both by keeping some nodes out of the queue, as the heuristic needs to be evaluated when a node is pushed into the queue.
Avoiding pushing low degree nodes into the queue is the focus of this section.

\subsection{Skip Degree Two Nodes}

We modify A* by processing low degree nodes consecutively without pushing them into the queue.
Our algorithm uses the undirected degree $d(x)$ of a node $x$.
Formally, $d(x)$ is the number of nodes $y$ such that $(x,y)\in E$ or $(y,x)\in E$.

Analogous to A*, our algorithm stores for every node $x$ a tentative distance $D[x]$.
Additionally, it maintains a minimum priority queue.
Diverging from A*, not all nodes can be pushed but every node has a tentative distance.

Our algorithm differs from A* when removing a node $x$ from the queue.
A* iterates over the outgoing arcs $(x,y)$ of $x$ and tries to reduce $D[y]$ by relaxing $(x,y)$.
If A* succeeds, $y$'s weight in the queue is set to $D[y]+h(y)$.
Our algorithm, however, behaves differently, if $d(y)\le 2$.
Our algorithm determines the longest degree two chain of nodes $x,y_1,\ldots, y_k, z$ such that $d(y_i)=2$ and $d(z) > 2$.
If our algorithm succeeds in reducing $D[y_1]$, it does not push $y_1$ into the queue.
Instead, it iteratively tries to reduce all $D[y_i]$.
If it does not reach $z$, then only $D$ is modified but no queue action is performed.
If $D[z]$ is modified and $d(z)>2$, $z$'s weight in the queue is set to $D[z]+h(z)$.

As the target node $t$ might have degree two, our algorithm cannot rely on stopping, when $t$ is removed from the queue.
Instead, our algorithm stops as soon as $D[t]$ is less than the minimum weight in the queue.

\subsection{Skip Degree Three Nodes}

In the previous section, we described an optimized A* variant that does not push degree two nodes.
In this section, we also avoid degree three nodes.

Denote by $x,y_1,\ldots, y_k, z$ a degree two chain as described in the previous section.
If $d(z) > 3$ or $z$ is in the queue, our algorithm proceeds as in the previous section.
Otherwise, there exist up to two degree chains $z,a_1,\ldots,a_p,b$ and $z,\alpha_1,\ldots,\alpha_q,\beta$ such that $a_1\neq y_k \neq \alpha_1$.
Our algorithm iteratively tries to reduce all $D[a_i]$ and $D[\alpha_i]$.
If it reaches $\beta$, $\beta$'s weight in the queue is set to $D[\beta]+h(\beta)$.
Analogously, if $b$ is reached, $b$'s weight is set to $D[b]+h(b)$.
If $b$ respectively $\beta$ are not reached, our algorithm does nothing.

\subsection{Stay in Largest Biconnected Component}

\label{sec:largested-biconnected-component}

A lot of nodes in road networks lead to dead-ends.
Unless the source or target is is in a dead-end, it is unnecessary to explore these nodes.

In the preprocessing phase, we compute the subgraph $G_C$, called \emph{core}.
$G_C$ is induced by the largest biconnected component of the undirected graph underlying $G$.
We do this using Tarjan's algorithm \cite{t-dfslg2-72}.
For every node $v$ in the input graph $G$, we store the attachment node $a_v$ to the core.
For nodes in the core, $a_v=v$.
We exploit that all attachment nodes are single node separators and the problem can be decomposed along them.

The query phase is divided into two steps.
In the first step, we apply A* with CH-Potentials to $G_C$ combined with the component that contains $s$.
This can be achieved implicitly by removing edges from $G_C$ into other components during preprocessing.
If $t$ is part of $G_C$ or in the same component as $s$, this A* search finds it.
Otherwise we find $a_t$.
In that case, we continue by searching a path from $a_t$ towards $t$ restricted to $t$'s biconnected component.
The final result is the concatenation of both paths.

\section{Algorithm Extensions}
\label{sec:extensions}

A* is a flexible algorithm.
Many extended problems can therefore be solved using the CH-Potentials algorithm.
In this section, we describe some extended problems and how they can be solved with CH-Potentials.

\subsection{Avoiding Tunnels and/or Highways}
\label{sec:no-tunnel-highway}

Avoiding tunnels and/or highways is a common feature of navigation devices.
Implementing this feature with CH-Potentials is easy.
We set $w_\ell$ to the freeflow travel time.
If an edge is a tunnel and/or a highway, we set $w_q$ to $+\infty$.
Otherwise, $w_q$ is set to the freeflow travel time.

\subsection{Forbidden Turns and Turn Costs}
\label{sec:no-turns}

The classical shortest path problem allows to freely change edges at nodes.
However in the real world, turn restrictions, such as a forbidden left or right turn, exist.
Such restrictions can be modeled using turn weights~\cite{gv-errnt-11,dgpw-crprn-13,bwzz-cchtc-20}.
% In this section, we first present the extended problem setting.
% Afterwards, we describe how it can be solved with CH-Potentials.
We extend CH-Potentials by using zero as lower bound for every turn weight in the heuristic.

% Ich bleibe bei v_i hier, damit keine Verwechselung mit den freien Variablen x,y,z gibt.
A \emph{turn weight} $w_t$  maps a pair of incident edges onto the turning time or $+\infty$.
A path with nodes $v_1, v_2,\ldots v_k$ has the following \emph{turn-aware weight}:\[
w_q(v_1, v_2) + \sum_{i=2}^{k-1}  w_t(v_{i-1},v_i,v_{i+1})  + w_q(v_i,v_{i+1})
\] The objective is to find a path between two given edges with minimum turn-aware weight.
The first term $w_q(v_1, v_2)$ is the same for all paths, as it only depends on the source edge.
It can thus be ignored during optimization.

A common strategy is to construct a \emph{turn-expanded} graph $G'$.
Edges in the input graph $G$ correspond to \emph{expanded nodes} in $G'$.
For every pair of incident edges $(x,y)$ and $(y,z)$ in $G$, there is an \emph{expanded edge} in $G'$ with expanded weight $w_t(x,y,z) + w_q(y,z)$.
A sequence of expanded nodes in the expanded graph $G'$ corresponds to a sequence of edges in the input graph $G$.
The weight of a path in $G'$ is equal to the turn-aware weight of the corresponding path in $G$ minus the irrelevant $w_q(v_1,v_2)$ term.
The turn-aware routing problem can be solved by searching for paths in $G'$.

The A* search runs in the expanded $G'$ and uses a heuristic $h'$.
The CH is constructed for $G$.
With CH-Potentials, we get the preprocessing-tight heuristic $h$.
We set $h'(x,y) = h(y)$.

To prove that $h'(x,y)$ is consistent, consider the turn-aware weight of a path $v_1,v_2\ldots v_k$ with $v_1=x$, $v_2=y$, $v_{k-1}=p$, and $v_k=q$.
We can lower bound the expression as follows:
\[
\sum_{i=2}^{k-1} w_\ell(v_i,v_{i+1}) \le \sum_{i=2}^{k-1} \underbrace{w_t(v_{i-1},v_i,v_{i+1})}_{0\le} + \sum_{i=2}^{k-1} \underbrace{w_q(v_i,v_{i+1})}_{w_\ell(v_i,v_{i+1})\le}
\]
$\sum_{i=2}^{k-1} w_\ell(v_i,v_{i+1})$ is a $yq$-path length in $G$.
It is no shorter than the shortest $yq$-path length in $G$, which is equal to $h(y)$.
The heuristic is therefore a lower bound.
It is consistent because it is derived from exact shortest paths in $G$ with $w_\ell$.

Sadly, the undirected graph underlying $G'$ is always biconnected, if the input graph is strongly connected.
The optimization described in Section~\ref{sec:largested-biconnected-component} is therefore ineffective.

With this setup CH-Potentials support turns without requiring turn information in the CH.

\subsection{Predicted Traffic or Time-Dependent Routing}
\label{sec:predicted-traffic}

The classical shortest path problem assumes that edge weights are scalars.
However in practice, travel times vary along an edge.
The primary reason is traffic.
Recurring traffic can be predicted by observing the traffic in the past.
It is common \cite{bgsv-mtdtt-13,bdpw-dtdrp-16,swz-sfert-19} to represent these predictions as \emph{travel time functions}.
An edge weight is no longer a scalar value but a function that maps the entry time onto the traversal time.
% Performing this prediction is outside of the scope of this paper.
It is common to refer to routing with predicted traffic as \emph{time-dependent routing}.
% Again, we first formalize the extended problem setting and then describe our solution with CH-Potentials.

Formally, the time-dependent query weight $w_p$ is a function from $E\times \mathbb{R}$ to $\mathbb{R}^+$.
$w_p(e, \tau)$ is the travel time through edge $e$ when entering it at moment $\tau$.
The input to the extended problem consists of a source node $s$ and a target node $t$, as in the classical problem formulation.
Additionally, the input contains a source time $\tau_s$.
A path with edges $e_1,e_2\ldots e_k$ is weighted using $\alpha_k$, which is defined recursively as follows:\[
\begin{split}
\alpha_{1} & = w_p(e_1, \tau_s) \\
\alpha_{k} & = \alpha_{k-1} + w_p(e_1, \alpha_{k-1})
\end{split}
\]
The objective is to find a path that minimizes $\alpha_k$.

If all travel time functions fulfill the \emph{no-waiting property}, this problem can be solved using a straight forward extension of Dijkstra's algorithm \cite{d-aassp-69}.
The necessary modification to A* is analogous.
Without the no-waiting property the problem is in general NP-hard \cite{or-tnp-89}.
The no-waiting property states that it is never beneficial to wait at a node before entering an edge.
Formally stated, the following must hold:\[
\forall e\in E,\tau\in \mathbb{R},\delta\in \mathbb{R}^+: w_p(e, \tau) \le w_p(e, \tau+\delta) + \delta
\]
Our implementation stores edge travel times using piece-wise linear functions.

To solve the time-dependent routing problem, we employ a strategy very similar to TD-ALT~\cite{ndls-bastd-12,dw-lbrdg-07}.
The main difference is that we use CH-Potentials instead of landmarks to guide the A* search.
We set $w_\ell(e) = \min_\tau w_p(e,\tau)$, that is the minimum travel time.
Following TD-ALT, we modify the A* search to use the tentative distance at a node $x$ as $\tau$ when evaluating the weight of an outgoing edge $(x,y)$.

With this setup, we extended CH-Potentials to support time-dependent routing.
As we were able to keep all travel time functions out of the CH, we avoid a lot of algorithmic complications compared to~\cite{bgsv-mtdtt-13,bdpw-dtdrp-16,swz-sfert-19}, which create shortcuts of travel time functions to combine hierarchical speedup techniques with time-dependent routing.

\subsection{Live and Predicted Traffic}
\label{sec:live-predicted-traffic}

Beside predicted traffic, we also consider live traffic.
Live traffic refers to the current traffic situation.
It is important to distinguish between predicted and live traffic.
The predicted travel time along an edge was estimated in the past.
It is possible that it differs from the current travel time significantly, if unexpected events happen.
Accidents are examples of such unexpected events.
Live traffic data is more accurate for the current moment than predicted data.
However, just using live traffic data is problematic for long routes as traffic changes while driving.
At some point, one wants to switch from live traffic to the predicted traffic.
In this section, we first describe a setup with only live traffic and then extend it with predicted traffic.

To support live traffic, we set $w_\ell$ to the freeflow travel time.
$w_q$ is set to the travel time accounting for current traffic.
As traffic only increases the travel time along an edge, $w_\ell$ is a valid lower bound for $w_q$.
This is all that is necessary to combine live traffic and CH-Potentials.

To combine live traffic with predicted traffic, we define a modified travel time function $w_q$ that is then used as query weights.
Denote by $w_p(e,\tau)$ the predicted travel time along edge $e$ at moment $\tau$.
Further, $w_c(e)$ is the travel time according to current live traffic.
Finally, we denote by $\tau_{\mathrm{soon}}$ the moment when we switch to predicted traffic.
In our experiments, we set $\tau_{\mathrm{soon}}$ to one hour in the future.
We need to make sure that the modified travel time function fulfills the no-waiting property.
For this reason, we cannot make a hard switch at $\tau_{\mathrm{soon}}$.
Our modified travel time function linearly approaches the predicted travel time. % with a bounded slope.
%
Formally, we set $w_q(e,\tau)$ to $w_c(e)$, if $\tau \leq \tau_{\mathrm{soon}}$.
Otherwise, we check whether $w_p(e,\tau_{\mathrm{soon}}) < w_c(e)$ is true.
If it is the case, we set $w_q(e,\tau)$ to $\max\{w_c(e)+(\tau_{\mathrm{soon}}-\tau), w_p(e,\tau)\}$.
Otherwise, we set $w_q(e,\tau)$ to $\min\{w_c(e)-(\tau_{\mathrm{soon}}-\tau), w_p(e,\tau)\}$.
We set $w_\ell$ again to the freeflow travel time.
We then run the algorithm of Section~\ref{sec:predicted-traffic} with $w_q$ as query weight.
In our implementation, we to not modify the representation of $w_p$ but evaluate the formulas above at each travel time evaluation.

With this setup, CH-Potentials support a combination of live and predicted traffic.
We did not make any modification, that would hinder a combination with other extensions.
Further adding tunnel and/or highway avoidance or turn-aware routing is simple.
This straight-forward integration of complex routing problems is the strength of the CH-Potentials algorithm.

\subsection{Temporary Driving Bans}

Truck routing differs from car routing due to night driving bans and other restrictions.
In~\cite{kswz-erptd-20}, a preliminary version of CH-Potentials is used for such a scenario.
The work considers time-dependent blocked edges and waiting at parking locations.
Further, a trade-off between arrival time and route quality is considered.

\section{Evaluation}

\label{sec:experiments}

\begin{table}
\centering
\caption{Instances used in the evaluation.}\label{tab:graphs}
\begin{tabular}{lrrrrrrr}
\toprule
 &                &                & \multicolumn{3}{c}{Preprocessing [s]} & \multicolumn{2}{c}{Aux. Data [MiB]} \\ \cmidrule(lr){4-6} \cmidrule(lr){7-8}
 & Vertices          & Edges          & \multirow{2}{*}{CH} & \multicolumn{2}{c}{CCH} & \multirow{2}{*}{CH} & \multirow{2}{*}{CCH} \\ \cmidrule(lr){5-6} & $[\cdot 10^6]$ & $[\cdot 10^6]$ &                     & Phase 1 & Phase 2 \\
\midrule
OSM Germany   &       16.2 &       35.4 &                          298.7 &        1\,467.4 &          10.1 &  645 & 616 \\
DIMACS Europe &       18.0 &       42.2 &                          276.2 &        2\,480.9 &          12.4 &  742 & 765 \\
TDGer06       &        4.7 &       10.8 &                           59.2 &           331.7 &           2.7 &  196 & 169 \\
TDEur17       &       25.8 &       55.5 &                          293.9 &        2\,102.3 &          14.1 & 1030 & 881 \\
TDEur20       &       28.5 &       60.9 &                          311.9 &        2\,219.5 &          15.2 & 1130 & 959 \\
\bottomrule
\end{tabular}

\end{table}

\begin{figure}
\centering
\includegraphics[width=\linewidth]{fig/scaled_weights.pdf}
\caption{
Running times on a logarithmic scale for queries on OSM Ger with scaled edge weights $w_q = \alpha \cdot w_\ell$.
The boxes cover the range between the first and third quartile.
The band in the box indicates the median, the diamond the mean.
The whiskers cover 1.5 times the inter quartile range.
All other running times are marked as outliers.
}\label{fig:scaled_weights}
\end{figure}

\begin{table}
\centering
\caption{Average query running times and number of queue pushs with different heuristics and optimizations on OSM Ger with $w_q = 1.05 \cdot w_\ell$.}\label{tab:building_blocks}
\begin{tabular}{lrrr}
\toprule
{} & Running Time &     Queue Pops &   Relaxed Arcs \\
{} &         [ms] & [$\cdot 10^3$] & [$\cdot 10^3$] \\
\midrule
No Deg2, No Deg3, No SCC, No Pot. &           1\,370.2 &          6\,947.2 &           19\,232.1 \\
No Deg2, No Deg3, No Pot.         &           1\,161.2 &          6\,126.2 &           17\,031.8 \\
No Deg3, No Pot.                  &           1\,046.6 &          5\,130.1 &           16\,240.9 \\
No Pot.                           &            810.8 &          2\,975.8 &           14\,852.6 \\
No Deg2, No Deg3, No SCC          &            433.1 &           914.8 &            2\,525.1 \\
No Deg2, No Deg3                  &            386.0 &           822.9 &            2\,278.1 \\
No Deg3                           &            332.2 &           697.6 &            2\,187.6 \\
                                  &            264.2 &           404.7 &            2\,025.7 \\
\bottomrule
\end{tabular}


\end{table}

\begin{table*}
\centering
\caption{
CH-Potentials performance for different route planning applications.
We report average running times and number of queue pushs.
We also report the average length increase, that is how much longer the final shortest distance is compared to the lower bound.
Finally, we report the average running time of Dijkstras algorithm as a baseline and the speedup over this baseline.
}\label{tab:applications}
\begin{tabular}{llrrrrr}
\toprule
 & &   Running &                Queue &     Length & Dijkstra & Speedup \\ & & time [ms] & pushs $[\cdot 10^3]$ & incr. [\%] &     [ms] &         \\
\midrule
OSM Ger & Unmodified &              0.6 &              0.5 &       0.0 &                    1\,922.6 &   3\,190.3 \\
        & Turns &              2.8 &              6.0 &       1.1 &                    4\,229.9 &   1\,522.1 \\
        & No Tunnels &             25.6 &             40.6 &       5.2 &                    1\,916.6 &     74.7 \\
        & No Highways &            330.2 &            501.9 &      42.1 &                    1\,759.3 &      5.3 \\
        & Live &            125.0 &            188.8 &      14.7 &                    1\,870.8 &     15.0 \\
        & TD &            356.1 &            285.8 &      28.3 &                    3\,285.2 &      9.2 \\
        & TD + Live &            374.8 &            308.0 &      33.0 &                    3\,305.8 &      8.8 \\
        & TD + Live + Turns &            853.7 &           1\,246.6 &      34.5 &                    6\,191.5 &      7.3 \\
\addlinespace
TDEur17 & TD &             89.6 &             81.1 &       3.9 &                    3\,385.6 &     37.8 \\
TDGer06 & TD &              4.4 &              6.4 &       3.1 &                     583.4 &    132.0 \\
\bottomrule
\end{tabular}


\end{table*}


In this section, we present our experimental evaluation.
Our benchmark machine runs openSUSE Leap 15.1 (kernel 4.12.14), and has 128\,GiB of DDR4-2133 RAM and an Intel Xeon E5-1630 v3 CPUs which has four cores clocked at 3.7\,Ghz and 4~$\times$~32\,KiB of L1, 8~$\times$~256\,KiB of L2, and 10\,MiB of shared L3 cache.
All experiments were performed sequentially.
Our code is written in Rust and compiled with rustc 1.47.0-nightly in the release profile. % with the target-cpu=native option.

\subparagraph{Inputs and Methodology}
Our main benchmark instance is a graph of the road network of Germany obtained from Open Street Map\footnote{\url{download.geofabrik.de/europe/germany-200101.osm.pbf}}.
To obtain the routing graph, we use the import from RoutingKit\footnote{\url{www.github.com/RoutingKit/RoutingKit}}.
The graph has 16M nodes and 35M edges.
For this instance, we have proprietary traffic data provided by Mapbox\footnote{\url{www.mapbox.com}}.
The  data includes a live traffic snapshot from Friday 2019/08/02 afternoon and comes in the form of 320K OSM node pairs and live speeds for the edge between the nodes.
It also includes traffic predictions for 38\% of the edges as predicted speeds for all five minute periods over the course of a week.
We exclude speed values which are faster than the freeflow speed computed by RoutingKit.
Additionally, we have two graphs with proprietary traffic predictions provided by PTV\footnote{\url{www.ptvgroup.com}}.
The PTV instances are not OSM-based.
One is an old instance of Germany with traffic predictions from 2006 for 7\% of the edges and the other one a newer instance of Europe with predictions for 27\% of the edges.
Table~\ref{tab:graphs} contains an overview over our instances.
In this table, we further include the sequential running time necessary to construct the CH.
We report preprocessing running times as averages over 10 runs.
For queries, we perform 10\,000 point-to-point queries where both source and target are nodes drawn uniformly at random and report average results.

\subparagraph{Experiments}
The first experiment varies the tightness of the heuristic.
The lower bound $w_\ell$ is set to the freeflow travel time.
The query weights $w_q$ are set to $\alpha \cdot w_\ell$, where $\alpha\ge 1$.
When $\alpha = 1$, the heuristic is tight.
Increasing $\alpha$ degrades the heuristic's quality.
Figure~\ref{fig:scaled_weights} depicts the results.
Clearly, $\alpha$ has significant influence on the running time.
Average running times range from below a millisecond to a few hundred milliseconds depending on $\alpha$.
Up to around $\alpha = 1.1$ the running time grows quickly.
After $\alpha = 1.1$ the growth slows down.
We observe that the running times for a fixed $\alpha$ vary strongly.
This is an interesting observation, as with uniform source and target sampling, nearly all queries are long-distance.
The query distance is thus not the reason.
After some investigation, we concluded that this is due to non-uniform road graph density.
Some regions have more roads per area than others.
The number explored A* nodes depends on the density of the search space area.
As the density varies, the running times vary.

Table~\ref{tab:building_blocks} shows the impact of our optimizations.
We run queries with three heuristics.
The Zero heuristic is simply $h(x)=0$ for all $x$.
This corresponds to using Dijkstra's algorithm.
Further, we run queries with our novel CH-Potential heuristic.
Finally, we use the unattainable lower bound Oracle-A* as described in Section~\ref{sec:oracle}.
Oracle-A* is an unbeatable theoretic lower bound.
The reported running time of Oracle-A* do \emph{not} account for the heuristic evaluation.
We include Oracle-A* as it is a lower bound not only for our algorithm but also related work.

We observe that the number of queue pushes roughly correlates with running time.
Each optimization reduces both queue pushes and running times.
% Further, each of our optimizations reduces the number of queue pushes.
Very interesting is the comparison with Oracle-A*.
Our algorithm is only 1.6 times slower than the unachievable theoretic lower bound.
This means that no competing algorithm such as ALT or CPD-Heuristics can be significantly faster than CH-Potentials.

Table~\ref{tab:applications} depicts the running times of CH-Potentials in the various applications, such as those described in Section~\ref{sec:extensions}.
Speedups compared to extensions of Dijkstra's algorithm for the respective application are depicted.
We start with the inflexible routing problem where $w_q = w_\ell$.
This is the problem variant solved by the basic CH algorithm.
In this setting, there is a huge speedup of 3243.
Such large speedups are typical for CH.
This shows that CH-Potentials gracefully converges to a CH in the $w_q = w_\ell$ special case.

In the other scenarios, the performance of CH-Potentials strongly depends on the quality of the preprocessing-tight heuristic.
We measure this quality using the length increase of $w_q$ compared to $w_\ell$.
Forbidding highways results in the largest length increase and in the smallest speedup.
The other extreme are turn restrictions.
They have only a small impact on the length increase.
The achieved speedups are therefore comparable to CH speedups.
%
Mapbox live traffic has a length increase of around 15\%, which yields running times of 127\,ms.
The length increase of Mapbox traffic predictions are about 18\%, and results in a running time of 200\,ms.
The speedup in the predicted scenario is larger than in the live setting, as the travel time function evaluations slow down Dijkstra's algorithm.
Combining predicted and live traffic results in a running time only slightly higher than for the predicted scenario.
Further adding turn restrictions, increases the running times.
This increase is mostly due to the BCC optimization of Section~\ref{sec:largested-biconnected-component} becoming ineffective when considering turns.
It is not due to the length increase of using turns.
With everything activated, our algorithm still has a speedup of 12.2 over the baseline.
Interestingly, the PTV traffic predictions have a much smaller length increase that the Mapbox predictions.
This results in smaller running times of our algorithm.

\section{Conclusion}
\label{sec:conclusion}

In this paper, we introduced CH-Potentials, a fast, exact, and flexible two-phase algorithm based on A* and CH for finding shortest paths in road networks.
The approach can handle a multitude of complex, integrated routing scenarios with very little implementation complexity.
The performance of CH-Potentials crucially depends on the availability of good lower bounds in the preprocessing phase.
Our experiments show, that this availability highly depends on the application.
We experimentally show that the running time of our algorithm is within a factor 1.6 of a hypothetical A* variant that can instantly evaluate the preprocessing-tight heuristic.
Achieving significantly smaller running times could still be possible in variations of the problem setting.
This leads to multiple avenues for future research.

Dropping the provable exactness requirement using a setup similar to anytime A* \cite{DBLP:conf/aaai/ZhouH02,DBLP:conf/nips/LikhachevGT03} would be interesting.
Another promising research avenue would be to investigate other graphs than road networks.
A lot of research into grid maps exists including a series of competitions called GPPC \cite{DBLP:conf/socs/SturtevantTTUKS15}.
Hierarchical techniques have been shown to work on these graphs \cite{DBLP:conf/aaai/UrasK14}.

%%
%% Bibliography
%%

%% Please use bibtex,

\pagebreak

\bibliography{dblp,references}
\bibliographystyle{aaai}
\end{document}
